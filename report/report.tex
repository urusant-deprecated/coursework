%%% Copyright (c) 2010, Илья w-495 Никитин
%%%
%%% Разрешается повторное распространение и использование как в виде исходного
%%% кода, так и в двоичной форме, если таковая будет получена, 
%%% с изменениями или без, при соблюдении следующих условий:
%%%
%%%     * При повторном распространении исходного кода должно оставаться
%%%       указанное выше уведомление об авторском праве, этот список условий и
%%%       последующий отказ от гарантий.
%%%     * Ни имя w-495, ни имена друзей или консультантов не могут быть
%%%       использованы в качестве поддержки или продвижения продуктов,
%%%       основанных на этом коде без предварительного письменного разрешения. 
%%%
%%% Этот код предоставлен владельцом авторских прав и/или другими
%%% сторонами "как она есть" без какого-либо вида гарантий, выраженных явно
%%% или подразумеваемых, включая, но не ограничиваясь ими, подразумеваемые
%%% гарантии коммерческой ценности и пригодности для конкретной цели. Ни в
%%% коем случае, если не требуется соответствующим законом, или не установлено
%%% в устной форме, ни один владелец авторских прав и ни одно  другое лицо,
%%% которое может изменять и/или повторно распространять программу, как было
%%% сказано выше, не несёт ответственности, включая любые общие, случайные,
%%% специальные или последовавшие убытки, вследствие использования или
%%% невозможности использования программы (включая, но не ограничиваясь
%%% потерей данных, или данными, ставшими неправильными, или потерями
%%% принесенными из-за вас или третьих лиц, или отказом программы работать
%%% совместно с другими программами), даже если такой владелец или другое
%%% лицо были извещены о возможности таких убытков.


\documentclass[unicode, 12pt, a4paper,oneside,fleqn]{article}
	%% Варианты []:
		% fleqn --- сдвигает формулы влево

	%% Варианты {}:
		% book
		% report
		% article
		% letter
		% minimal (???)

\usepackage{styles/main} 
	% подключаем набор стилей 

\ifpdf
	\hypersetup{ 
			pdffitwindow=false,
			pdfstartview={FitH},			
		pdftitle={Это шаблонный документ v0.92}, 
		pdfauthor={Илья w-495 Никитин}, 
		pdfcreator={LaTeX2e + TexMakerX}, 
		pdfsubject={Тема}, 
		pdfproducer={Илья w-495 Никитин}, 
		pdfkeywords={Шаблон}
	}
\fi

\begin{document}

%%%%%%%%%%%%%%%%%%%%%%%%%%%%%%%%%%%%%%%%%%%%%%%%%%%%%%%%%%%%%%%%%%%%%%%%%%%%%%%%
%%%
%%% бесполезное содержимое
%%%

	\input{src/work-titlepage} 	% титульный лист
	\pagebreak
	\section{Введение}

Одной из самых широко встречающихся на данный момент задач в анализе текстов является задача кластеризации, то есть разбиение некоторого множества документов на такие группы, что документы из одной группы так или иначе похожи друг на друга, в то время как документы из разных групп совершенно разные.

Одним из примеров такой задачи является кластеризация новостных потоков, где в качестве текстов выступают новостные статьи за какой-то период времени. Это позволяет выделить из большого количества документов основные сюжеты, что значительно облегчает поиск актуальной информации. Кроме того, изучив различные документы одного кластера, мы можем проанализировать статьи об одном и том же событии из разных источников и выделить общие и различающиеся моменты, что позволяет сделать получаемую информацию более достоверной.

Другой пример, описанный также в 16й главе из [1], - информационный поиск. Здесь широко применяется \emph{кластерная гипотеза}, которая гласит, что документы из одного кластера ведут себя примерно одинаково при одном и том же поисковом запросе. Это позволяет лучше подбирать документы для каждого запроса, поскольку если мы уже определили релевантность некоторой статьи запросу, то в соответствии с кластерной гипотезой другие документы из того же кластера тоже релевантны этому запросу.

Кроме того, кластеризация хорошо помогает, если пользователь ищет не какую-то конкретную информацию, а просто хочет почитать что-нибудь, что было бы ему интересно. В этом случае поиск интересных статей можно осуществить следующим образом. Изначально все множество статей кластеризуется на небольшое количество кластеров, и для каждого кластера определяется общая тематика статей в этом кластере. Далее пользователь выбирает одну или несколько (но не все) темы, которые ему интересны. Полученные документы снова кластеризуются. Это повторяется до тех пор, пока множество документов не станет достаточно маленьким. В таком случае пользователю просто предоставляется список полученных статей, которые с достаточно высокой вероятностью будут интересны пользователю. В частности, похожая идея реализована в сервисе dmoz.org.
	\section{Нахождение попарной похожести текстов}

\subsection{Понятие об аннотированном суффиксном дереве}
Чтобы определить аннотированное суффиксное дерево, определим сначала более общее понятие \emph{бора}. Итак, \emph{бор} - это структура данных для хранения набора строк, представляющая собой корневой дерево, каждому ребру которого соответствует некоторый символ. При этом некоторые вершины помечены как \emph{терминальные}. Каждой вершине ставится в соответствие строка как последовательность символов, написанных на ребрах между корнем и этой вершиной. Говорят, что бор \emph{принимает} строку $s$, если есть такая терминальная вершина, которой соответствует строка $s$.\newline
Приведем пример. Пусть у нас есть набор строк, например, A = \{"гол"\,, "горн"\,, "гик"\,, "пик"\}. Для него бор будет выглядеть следующим образом:
\begin{center} 
	\includegraphics[width=10cm]{img/trie}
\end{center}
Здесь для удобства все вершины пронумерованы, а терминальные отмечены красным цветом.\newline
Тогда, например, вершине 10 соответствует строка "горн"\,, поскольку последовательность ребер от корня до нее выглядит как $0 \rightarrow 1, 1 \rightarrow 3, 3 \rightarrow 6, 6 \rightarrow 10$, и ей соответствует последовательность символов "горн". \newline
Заметим, что в текущем определении есть некоторая неоднозначность. Действительно, например, следующее дерево также удовлетворяет всем условиям:
\begin{center} 
	\includegraphics[width=10cm]{img/trie2}
\end{center}
Чтобы избежать этого, потребуем дополнительно, чтобы для каждой вершины $v$ и символа $c$ было не более одного ребра из $v$ в какого-нибудь из ее потомков, соответствующего символу $c$. Кроме того, потребуем, чтобы все листья были терминальными вершинами, то есть соответствовали какой-нибудь строке из множества (иначе дерево может быть сколь угодно большим). Тогда для любого множества строк бор определяется однозначно. Действительно, пусть есть 2 различных бора для одного и того же множества строк. Найдем ребро, которое есть в одном, но нету во втором, причем если таких несколько, то выберем то, которое ближе всех к корню. Пусть это ребро $v \rightarrow to$, ему соответствует символ $c$, а вершине $v$ - строка $s$. Поскольку каждый лист - терминальная вершина, какой-то потомок $to$ - терминальная вершина, значит, в множестве строк есть какая-то строка $S$ с префиксом $s + c$. С другой стороны, если мы пойдем от корня по строке $s$ во втором дереве (то есть сначала пойдем по ребру, которому соответствует символ $s_0$, потом - $s_1$, и так далее), то мы, очевидно, придем в ту же вершину $v$. Однако, мы не сможем пойти дальше по символу $c$ по предположению, значит, второй бор не может принимать строку с префиксом $s + c$. Таким образом, мы пришли к противоречию, и тем самым доказали, что бор по множеству строк определяется однозначно. Кроме того, доказательство показывает способ построения этого бора. Пусть изначально бор пустой, то есть состоит из 1й вершины - корня. Будем последовательно добавлять в него строки следующим образом: будем идти по имеющемуся дереву в соответствии с очередной строкой $s$, и если в какой-то момент не находим нужного ребра, то просто создаем новое ребро, ведущее из текущей вершины в новую и соответствующее нужному нам символу. В конце помечаем вершину, в которой оказались, терминальной. Очевидно, этот алгоритм не может добавить 2 ребра с одним и тем же символом из одной вершины, потому что мы создаем ребро, только если его еще нет, и каждый лист будет терминальным, потому что каждая вершина соответствует некоторому префиксу какой-то из строк множества, значит, у каждой вершины есть терминальный потомок. \newline
Пусть теперь у нас есть некоторый текст, который мы будем рассматривать как некоторый набор слов. \emph{Суффиксным деревом} для этого текста назовем бор, который принимает суффиксы всех слов и только их. \newline
Снова приведем пример. Пусть есть, например, текст, состоящий из 2х слов "голод"\ и "холод". Тогда суффиксное дерево должно принимать все суффиксы этих двух слов, то есть строки "голод"\,, "олод"\,, "лод"\,, "од"\,, "д"\,, "холод". Построим бор на этих строках по выше приведенному алгоритму:
\begin{center} 
	\includegraphics[width=10cm]{img/suffix_tree}
\end{center}

Теперь введем понятие \emph{аннотированного суффиксного дерева}. Аннотированное суффиксное дерево для набора строк - это суффиксное дерево, в котором для каждой вершины дополнительно хранится целое число (частота) - количество строк из набора, префиксом которых является строка, соответствующая этой вершине. Заметим, что это не всегда совпадает с количеством терминальных вершин в поддереве, поскольку в наборе каждая из строк может встречаться несколько раз. \newline
Например, для суффиксного дерева, которое было построено ранее, аннотированное суффиксное дерево будет выглядеть следующим образом:
\begin{center} 
	\includegraphics[width=10cm]{img/annotated_suffix_tree}
\end{center}
Так, в этом примере суффиксы "олод"\,, "лод"\,, "од"\,, "д"\ учитывыются по 2 раза. \newline
Отметим, алгоритм, приведенный выше для построения суффиксного дерева легко модифицировать для подсчета частот. Действительно, при добавлении очередного суффикса в дерево значения частот в вершинах пути от корня к вершине, отвечающей за этот суффикс, увеличиваются на 1, а в остальных - не изменяются. Поэтому можно просто добавлять 1 к частоте каждой вершины, которую проходим. У только что созданных вершин это значение должно быть равным 0, поскольку мы еще ни разу не проходили эту вершину.\newline
\subsection{Нахождение величины похожести текстов на основе суффиксных деревьев}
Пусть есть 2 текста $A, B$. Построим по ним аннотированные суффиксные деревья $S_A, S_B$. Далее найдем для этих суффиксных деревьев \emph{общее поддерево} $S$, то есть такое подмножество вершин, которые есть в обоих деревьях. При этом в качестве частот в $S$ будем использовать $f(a, b)$, где $a, b$ - частоты вершины в $S_A, S_B$, а $f$ - некоторая функция. Теперь остается оценить это дерево. В качестве оценки используем величину $\sum\limits_v \frac{frequency(v)}{frequency(p(v))}$, где $frequency(v)$ - частота вершины, а $p(v)$ - предок вершины v. Таким образом, мы суммируем по всем вершинам их условную вероятность появления в тексте, то есть вероятность перехода в вершину $v$ при условии, что мы уже дошли до $p(v)$. \newline
Рассмотрим, например, 2 текста A = \{"голод"\,, "холод"\}, B = \{"солод"\,, "вол"\}. Для $A$ мы уже строили дерево выше, для $B$ оно выглядит следующим образом:
\begin{center} 
	\includegraphics[width=10cm]{img/suffix_tree2}
\end{center}
Тогда общее поддерево имеет следующий вид:
\begin{center} 
	\includegraphics[width=10cm]{img/common_suffix_tree}
\end{center}
Пусть $f(a, b) = \frac{a + b}{2}$. Тогда оценка похожести равна:
$$ \frac{f(3, 4)}{f(8, 10)} + \frac{f(1, 2)}{f(8, 10)} + \frac{f(2, 2)}{f(8, 10)} + \frac{f(1, 2)}{f(3, 4)} + \frac{f(2, 2)}{f(3, 4)} + \frac{f(1, 2)}{f(2, 2)} + \frac{f(1, 2)}{f(2, 2)} + \frac{f(1, 2)}{f(1, 2)} + \frac{f(1, 2)}{f(1, 2)} = $$
$$\frac{3 + 4}{8 + 10} + \frac{1 + 2}{8 + 10} + \frac{2 + 2}{8 + 10} + \frac{1 + 2}{3 + 4} + \frac{2 + 2}{3 + 4} + \frac{1 + 2}{2 + 2} + \frac{1 + 2}{2 + 2} + \frac{1 + 2}{1 + 2} + \frac{1 + 2}{1 + 2} = 5.27778$$
\subsection{Параметризация}
Как можно заметить, данный алгоритм нахождения похожести принимает один дополнительный параметр, а именно, функцию $f(a, b)$. В данной работе были использованы следующие функции:
\begin{itemize}
\item $f(a, b) = \frac{a + b}{2}$
\item $f(a, b) = min(a, b)$
\item $f(a, b) = max(a, b)$
\item $f(a, b) = \sqrt{ab}$
\end{itemize}
	\section{Кластеризация}
\subsection{Оценка эффективности кластеризации}
Чтобы иметь возможность сравнивать различные алгоритмы кластеризации, нужно привести способ оценивания их эффективности. В данной работе для тестирования использовалось подмножество популярной коллекции Reuters-21578, а именно, были выбраны все тексты из нее, которые принадлежат ровно одной из списка категорий \{acq, crude, earn, grain, interest, money-fx, ship, trade\}. Таким образом, выделялось 8 кластеров.
Для каждого из тестируемых алгоритмов было и для каждого варианта функции из списка выше находится кластеризация, то есть разбиение множества документов на несколько групп, обозначенных метками. Кроме того, была также использована 5я функция, считающаяся стандартной для подобных задач. Она вводится с целью сравнить предлагаемый алгоритм с уже существующими. Далее, с целью вычисления эффективности решения для каждого из найденных кластеров находится самая часто встречающаяся категория, которая и полагается основной для данного кластера, и затем считается доля ошибок, то есть количество документов, категория которых не совпадает с категорией их кластера, деленное на общее количество документов. \newline
Теперь рассмотрим алгоритмы, использованные для кластеризации, а именно, следующие:
\begin{itemize}
\item Модификация k-means
\item K-medoids
\item Spectral clustering
\end{itemize}
Отметим, что в первых 2х алгоритмах требуется не похожесть, а, наоборот, расстояние между документами, то есть некоторая величина, которая тем больше, чем меньше документы похожи. В качестве такой величины удобно использовать значение $\frac{1}{similarity}$.
\subsection{Модификация k-means}
Классический k-means заключается в следующем. Пусть есть сколько-то точек в некотором линейном пространстве. Сначала выбираем случайно k точек - центроиды наших кластеров. Далее итеративно выполняем следующее:
\begin{itemize}
\item Пересчитываем кластеры, а именно, для каждой точки множества находим ближайший центроид.
\item Пересчитываем центроиды, то есть для каждого из полученных кластеров находим новый центроид как центр масс кластера.
\end{itemize}
Однако, проблема заключается в том, что у нас нет самих координат точек, а только расстояния между ними (расстояния считаются как $\frac{1}{similarity}$). В связи с этим модифицируем алгоритм следующим образом: будем выбирать изначальные центроиды как элементы нашего множества, а при пересчете будем выбирать тот элемент множества, от которого сумма расстояний до точек кластера минимальна. Таким образом, нам не требуется знать координаты точек, а только попарные расстояния между ними. \newline
Ниже приведены результаты работы этого алгоритма для каждого способа нахождения похожести текстов при 60 итерациях.
\begin{itemize}
\item Для $f(a, b) = \frac{a + b}{2}$ результат равен $0.523162$.
\item Для $f(a, b) = min(a, b)$ результат равен $0.469608$.
\item Для $f(a, b) = max(a, b)$ результат равен $0.516789$.
\item Для $f(a, b) = \sqrt(ab)$ результат равен $0.514338$.
\item Для стандартного способа результат равен $0.655392$.
\end{itemize}

\subsection{K-medoids}
Идея алгоритма похожа на k-means, но отличается способом пересчета центроидов. Изначально мы инициализируем кластеризацию, выбрав k случайных центроидов и отнеся каждый из остальных элементов в кластер к ближайшему центроиду. Далее, на каждой итерации для каждого документа пробуем поменять его с центром его кластера. Для полученной конфигурации пересчитываем кластеры (т.е. для каждого документа выбираем ближайший центр) и вычисляем некоторую величину, которая показывает, насколько хороша полученная кластеризация. В данном случае удобно использовать сумму расстояний от каждого документа до его центра. Таким образом, чем меньше эта величина, тем лучше кластеризация.
Итак, для каждой из таких замен мы посчитали, насколько улучшится (и улучшится ли вообще) кластеризация. Теперь просто выберем ту, которая дает самое большое улучшение, и применим. \newline
Количество итераций в этом алгоритме, в отличие от k-means, не выбирается фиксированным, а алгоритм выполняется до тех пор, пока у нас получается найти улучшающую замену. При выбранных ограничениях (~8000 элементов) на практике требуется всего несколько десятков итераций, поэтому алгоритм достаточно быстро завершает работу.
Ниже приведены результаты работы этого алгоритма так же, как это было сделано в предыдущем пункте.
\begin{itemize}
\item Для $f(a, b) = \frac{a + b}{2}$ результат равен $0.545833$.
\item Для $f(a, b) = min(a, b)$ результат равен $0.484681$.
\item Для $f(a, b) = max(a, b)$ результат равен $0.575$.
\item Для $f(a, b) = \sqrt(ab)$ результат равен $0.546569$.
\item Для стандартного способа результат равен $0.648284$.
\end{itemize}
\subsection{Spectral clustering}
Этот алгоритм был выбран как самый подходящий из модуля scikit-learn для python. Его суть состоит в нахождении координат точек в k-мерном пространстве по матрице похожести, и после этого применяется k-means к полученным точкам.
Ниже приведены результаты работы этого алгоритма.
\begin{itemize}
\item Для $f(a, b) = \frac{a + b}{2}$ результат равен $0.6867647058823529$.
\item Для $f(a, b) = min(a, b)$ результат равен $0.6938725490196078$.
\item Для $f(a, b) = max(a, b)$ результат равен $0.6843137254901961$.
\item Для $f(a, b) = \sqrt(ab)$ результат равен $0.6922794117647059$.
\item Для стандартного способа результат равен $0.6931372549019608$.
\end{itemize}

	\section{Выводы}
На основе проделанного исследования можно сделать несколько выводов. \newline
Во-первых, доля ошибок примерно одинаковая для каждой из четырех функций, которые выбирались как параметры при нахождении похожести текстов. Это не значит, что можно выбирать совершенно любую функцию от двух переменных: все приведенные примеры удовлетворяют условию $min(a, b) \leq f(a, b) \leq max(a, b)$, что логично. Однако, можно сделать вывод, что выбор функции не является существенным, и поэтому в большинстве случаев можно использовать какую-нибудь достаточно простую, например, среднее арифметическое. \newline
Во-вторых, результаты говорят, что в первых двух алгоритмах стандартный подход к нахождению похожести значительно более эффективен, однако в Spectral clustering они различаются лишь незначительно, что означает возможность применения этого подхода в реальных задачах. Тем не менее, в текущей реализации это не слишком хорошая идея, поскольку генерация таблицы занимает достаточно много времени, соответственно, для использования этого алгоритма его необходимо значительно оптимизировать. \newline

	%%%%%%%%%%%%%%%%%%%%%%%%%%%%%%%%%%%%%%%%%%%%%%%%%%%%%%%%%%%%%%%%%%%%%%%%%%%%%%%%
	%%%
	%%% дополнительное (свое) задание верхнего колонтитула
	%%% 
	%%%
	%	\makeatletter
	%	\renewcommand{\@oddhead}{ \textcolor{blue}{Лекция (задача) \arabic{lections}} \hfil \par
	%	\hfil  \leftmark \hfil \rightmark }
	%	\makeatother
%& -shell-escape
	
%%%%%%%%%%%%%%%%%%%%%%%%%%%%%%%%%%%%%%%%%%%%%%%%%%%%%%%%%%%%%%%%%%%%%%%%%%%%%%%%
%%%
%%% полезное содержимое
%%%

	% пример %%%%%%%%%%%%%%%%%%%%%%%%%%%%%%%%%%%%%%%%%%%%%%%%%%%%%%%%%%%%%%%%%%%
	
		%\input{\SRC/-example-about}
		%\input{\SRC/-example-sources}
		%\input{\SRC/-example-pictures}
		%\input{\SRC/-example-plot}
		%\input{\SRC/-example-text}
	
	% лекции %%%%%%%%%%%%%%%%%%%%%%%%%%%%%%%%%%%%%%%%%%%%%%%%%%%%%%%%%%%%%%%%%%%
	
	%	\input{\SRC/lct-01} %% лекция #1
	
	% лабы\курсовые %%%%%%%%%%%%%%%%%%%%%%%%%%%%%%%%%%%%%%%%%%%%%%%%%%%%%%%%%%%%%%%%%%%
	
	%	\input{\SRC/work-problem} 		%% постановка
	%	\input{\SRC/work-theory} 		%% теоретическая часть
	%	\input{\SRC/work-solution} 		%% решение
	%	\input{\SRC/work-example} 		%% примеры
	%	\input{\SRC/work-conclusions} 	%% выводы
		
\end{document}

%%
%%
%%

