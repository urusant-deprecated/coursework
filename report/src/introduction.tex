\section{Введение}

Одной из самых широко встречающихся на данный момент задач в анализе текстов является задача кластеризации, то есть разбиение некоторого множества документов на такие группы, что документы из одной группы так или иначе похожи друг на друга, в то время как документы из разных групп совершенно разные.

Одним из примеров такой задачи является кластеризация новостных потоков, где в качестве текстов выступают новостные статьи за какой-то период времени. Это позволяет выделить из большого количества документов основные сюжеты, что значительно облегчает поиск актуальной информации. Кроме того, изучив различные документы одного кластера, мы можем проанализировать статьи об одном и том же событии из разных источников и выделить общие и различающиеся моменты, что позволяет сделать получаемую информацию более достоверной.

Другой пример, описанный также в 16й главе из [1], - информационный поиск. Здесь широко применяется \emph{кластерная гипотеза}, которая гласит, что документы из одного кластера ведут себя примерно одинаково при одном и том же поисковом запросе. Это позволяет лучше подбирать документы для каждого запроса, поскольку если мы уже определили релевантность некоторой статьи запросу, то в соответствии с кластерной гипотезой другие документы из того же кластера тоже релевантны этому запросу.

Кроме того, кластеризация хорошо помогает, если пользователь ищет не какую-то конкретную информацию, а просто хочет почитать что-нибудь, что было бы ему интересно. В этом случае поиск интересных статей можно осуществить следующим образом. Изначально все множество статей кластеризуется на небольшое количество кластеров, и для каждого кластера определяется общая тематика статей в этом кластере. Далее пользователь выбирает одну или несколько (но не все) темы, которые ему интересны. Полученные документы снова кластеризуются. Это повторяется до тех пор, пока множество документов не станет достаточно маленьким. В таком случае пользователю просто предоставляется список полученных статей, которые с достаточно высокой вероятностью будут интересны пользователю. В частности, похожая идея реализована в сервисе dmoz.org.