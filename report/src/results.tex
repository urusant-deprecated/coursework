\section{Выводы}
На основе проделанного исследования можно сделать несколько выводов. \newline
Во-первых, доля ошибок примерно одинаковая для каждой из четырех функций, которые выбирались как параметры при нахождении похожести текстов. Это не значит, что можно выбирать совершенно любую функцию от двух переменных: все приведенные примеры удовлетворяют условию $min(a, b) \leq f(a, b) \leq max(a, b)$, что логично. Однако, можно сделать вывод, что выбор функции не является существенным, и поэтому в большинстве случаев можно использовать какую-нибудь достаточно простую, например, среднее арифметическое. \newline
Во-вторых, результаты говорят, что в первых двух алгоритмах стандартный подход к нахождению похожести значительно более эффективен, однако в Spectral clustering они различаются лишь незначительно, что означает возможность применения этого подхода в реальных задачах. Тем не менее, в текущей реализации это не слишком хорошая идея, поскольку генерация таблицы занимает достаточно много времени, соответственно, для использования этого алгоритма его необходимо значительно оптимизировать. \newline